\c Version $Id$
\node CGC Domain
\chapter{CGC Domain}

\Author{J. Buck \\
S. Ha \\
E. A. Lee \\
T. M. Parks}
\date{1/19/94}

The \dfn{CGC domain} of Ptolemy generates code
for the \code{C} programming language.
\cindex{code generation in C}
\cindex{C code generation}
\cindex{CGC domain}
This domain supports both synchronous dataflow
(\pxref{SDF Domain})
\cindex{synchronous dataflow}
\cindex{dataflow, synchronous}
and boolean-controlled dataflow
(\pxref{BDF Domain})
\cindex{boolean-controlled dataflow}
\cindex{dataflow, boolean-controlled}
models of computation.
The model of computation associated with a particular program graph
is determined by which target is selected.
The \code{bdf-CGC} target supports the BDF model, while all other targets
in the CGC domain support the more limited SDF model.
Code can be generated for single-processor machines and multi-processor
machines.
The targets that support single-processors include \code{default-CGC},
\code{TclTk_Target}, and \code{bdf-CGC}.
The multi-processor target is \code{unixMulti_C}.

For an overview of CGC stars and demos, please refer to the star atlas.

\node CGC Single-Processor Targets
\section{Single-Processor Targets}

The single-processor targets of the CGC domain generate \code{C} code
from dataflow program graphs.
Code generation is controlled by the \var{host}, \var{directory}, and
\var{file} paramters as described in \nxref{CG Domain}.
\vrindex{host, parameter}
\vrindex{directory, parameter}
\vrindex{file, parameter}
The command used to compile the code is determined by the \var{compileCommand},
\var{compileOptions}, and \var{linkOptions} parameters.
Compilation and execution are controlled by the \var{display?},
\var{compile?}, and \var{run?} parameters, also described in
\nxref{CG Domain}.
The parameters common to the CGC targets are listed below.

\begin{statelist}
\state{INT}{staticBuffering}{TRUE}
If \samp{TRUE}, then attempt to static, compile-time addressing of data
buffers between stars.
Otherwise, use dynamic, run-time addressing.

\state{STRING}{funcName}{main}
The name of the main function.
The default value of \samp{main} is suitable for generating standalone
programs.
Choose another name if you wish to use the generated code as a procedure
that is called from your own main program.

\state{STRING}{compileCommand}{cc}
Command name of the compiler.

\state{STRING}{compileOptions}{}
Options passed to the compiler.

\state{STRING}{linkOptions}{-lm}
Options passed to the linker.

\state{STRINGARRAY}{resources}{STDIO}
List of abstract resources that the host computer has.
\end{statelist}

\node default-CGC
\subsection{\protect\code{default-CGC}}

The \code{default-CGC} target generates \code{C} code for a single computer
from a SDF program graph.

\node TclTk_Target
\subsection{\protect\code{TclTk_Target}}

This target, which is derived from the \code{default-CGC} target
(\pxref{default-CGC}), must be used when Tcl/Tk stars are present in
the program graph.
The parameters differ from those of the \code{default-CGC} target only
in their default values.

\begin{statelist}
\state{INT}{loopingLevel}{1}

\state{STRING}{funcName}{go}

\state{STRING}{compileOptions}{-I$PTOLEMY/tcl/tk/src \\
    -I$PTOLEMY/tcl/tcl/src \\
    -I$PTOLEMY/src/domains/cgc/tcltk/lib}

\state{STRING}{linkOptions}{-L$PTOLEMY/tcl/tk/lib/$ARCH.opt \\
    -L$PTOLEMY/tcl/tcl/lib/$ARCH.opt \\
    -L$PTOLEMY/lib.$ARCH \\
    -L$PTOLEMY/tcl/tk/lib/$ARCH.opt \\
    -L/usr/X11/lib -ltk -ltcl -lptk -lXpm -lX11 -lm}

\end{statelist}

\node bdf-CGC
\subsection{\protect\code{bdf-CGC}}

This target supports the boolean-controlled dataflow (BDF) model of
computation.
It must be used when BDF stars are present in the program graph.
It can also be used with program graphs that contain only synchronous
dataflow (SDF) stars.

The \code{bdf-CGC} target has the same parameters as the \code{default-CGC}
target (\pxref{default-CGC})
with the exception that the \var{Looping Level} parameter is absent.

\node CGC Multi-Processor Targets
\section{Multi-Processor Targets}

\node unixMulit_C
\subsection{\protect\code{unixMulti_C}}

\node CGCcm5
\subsection{\protect\code{CGCcm5}}

\node CGC Star Overview
\section{An Overview of CGC Stars}

\begin{figure}
\begin{center}
\ \psfig{figure=cgc-stars.ps}
\end{center}
\end{figure}

\node CGC Source Stars
\subsection{Source Stars}

\begin{description}
\item[Const]
\item[Ramp]
\item[IIDUniform]
\item[WaveForm]
\item[PCMread]
\item[singen]
\item[TkSlider]
\item[TkEntry]
\item[TkImpulse]
\end{description}

\node CGC Sink Stars
\subsection{Sink Stars}

\begin{description}
\item[BlackHole]
\item[Xgraph]
\item[XMgraph]
\item[Xscope]
\item[XYgraph]
\item[Printer]
\item[PCMwrite]
\item[SunSound]
\end{description}

\node CGC Arithmetic Stars
\subsection{Arithmetic Stars}

\node CGC Nonlinear Stars
\subsection{Nonlinear Stars}

\node CGC Control Stars
\subsection{Control Stars}

\node CGC Conversion Stars
\subsection{Conversion Stars}

\node CGC Signal Processing Stars
\subsection{Signal Processing Stars}

\node CGC Dynamic Dataflow Stars
\subsection{Dynamic Dataflow Stars}

\node CGC Boolean-Controlled Dataflow Stars
\subsection{Boolean-Controlled Dataflow Stars}

\node CGC Tcl/Tk Stars
\subsection{Tcl/Tk Stars}

\node CGC Demo Overview
\section{An Overview of CGC Demos}

\begin{figure}
\begin{center}
\ \psfig{figure=cgc-demo.ps}
\end{center}
\caption{The top-level palette for CGC demos.}
\end{figure}
