% Version $Id$
\chapter{CGC Domain}
\label{CGC Domain}

\Author{Joseph T. Buck \\
Soonhoi Ha \\
Edward A. Lee \\
Thomas M. Parks}
\index{Buck, J. T.}
\index{Ha, S.}
\index{Lee, E. A.}
\index{Parks, T. M.}

\Contrib{Kennard White}
\index{White, K.}
%\date{$Date$}

The \dfn{CGC domain} generates code for the \code{C} programming
language.  Chapter~\ref{Code Generation} describes the features
common to all code generation domains.
The basic principles of writing code generation stars are explained in
\link{section}[~\Ref]{Writing code generation stars}.  You will find explanations for
codeblocks, macros, and attributes there.
This chapter explains features
specific to the CGC domain.  Refer to the CGC domain chapter in the
user's manual for an introduction to this domain.

\begin{ifhtml}
% The P_GLOBAL and P_LOCAL attributes are not used at this time.

\section{Attributes}
\label{CGC Attributes}

The attributes \code{P\_GLOBAL} and \code{P\_LOCAL} are defined for use
with the \code{CGCPortHole}\index{CGCPortHole class} class.  They are
the counterparts of the \code{A\_GLOBAL}\index{A\_GLOBAL attribute} and
\code{A\_LOCAL}\index{A\_LOCAL attribute} attributes for states.  They
control whether the buffer associated with a port is declared as a
global variable or a local variable.

\begin{indexlist}{ attribute}

\variable{const Attribute}{P\_GLOBAL}
Declare the buffer associated with a port as a global variable, which
can be accessed outside the main function.

\variable{const Attribute}{P\_LOCAL}
Declare the buffer associated with a port as a local variable within the
main function.

\end{indexlist}

By default, all
buffers are declared as local variables within the main function.  This
allows compilers to make optimizations that would not be possible if
the buffers had been declared as global variables.  If the buffers must
be accessed outside the main function, then use the \code{P\_GLOBAL}
attribute as shown in the following example.

\begin{example}
output
\{
    name \{ out \}
    type \{ float \}
    attributes \{ P\_GLOBAL \}
\}
\end{example}

\end{ifhtml}

\section{Code Generation Methods}
\label{CGC Code Generation Methods}

The \code{addCode} method is context sensitive so that it will ``do the
right thing'' when invoked from within the \code{initCode}, \code{go},
and \code{wrapup} methods of \code{CGCStar}.  Refer to
\link{section}[~\Ref]{Writing code generation stars}
for documentation on \code{addCode}, including
context sensitive actions and conditional code generation.  There are
several additional code-generation methods defined in the CGC domain.
The \code{addInclude} method is used to generate \samp{\#include
\var{file}} directives.  The \code{addDeclaration} method is used to
declare local variables within the main function.  The \code{addGlobal}
method is used to declare global variables outside the main function.
As with \code{addCode}, these methods return \samp{TRUE} if code was
generated for the appropriate stream and \samp{FALSE} otherwise.  These
methods are member functions of the \code{CGCStar}\index{CGCStar
class} class.

\begin{indexlist}{CGCStar}

\method{int}{addInclude}{(const char* \var{file})}
Generate the directive \samp{\#include \var{file}} in the
\code{include}\index{include!CGCTarget member} stream.  The string
\var{file} must include quotation marks ( ``name'' ) or angle brackets
( <name> ) around the name of the file to be included.  Only one
\samp{\#include \var{file}} directive will be generated for the file,
even if \code{addInclude} is invoked multiple times with the same
argument.  Return \samp{TRUE} if a new directive was generated.

\method{int}{addDeclaration}{(const char* \var{text}, const char* \var{name} = NULL)}
Add \var{text} to the \code{mainDecls}\index{mainDecls!CGCTarget
member} stream.  Use \var{name} as the identifying key for the code
fragment if it is provided, otherwise use \var{text} itself as the
key.  Code will be added to the stream only the first time that a
particular key is used.

\method{int}{addGlobal}{(const char* \var{text}, const char* \var{name} = NULL)}
Add \var{text} to the \code{globalDecls}\index{globalDecls!CGCTarget
method} stream.  Use \var{name} as the identifying key for the code
fragment if it is provided, otherwise use \var{text} itself as the
key.  Code will be added to the stream only the first time that a
particular key is used.

\begin{ifhtml}
% addCode will add to the mainInit stream when called from initCode.

\method{int}{addMainInit}{(const char* \var{text}, const char* \var{name} = NULL)}
Add \var{text} to the \code{mainInit}\index{mainInit!CGCTarget member}
stream.  If \var{name} is not \samp{NULL}, then use it as the
identifying key for the code fragment, which is added conditionally.

\end{ifhtml}

\end{indexlist}

The following streams, which are used by the code generation methods
just described, are defined as members of the
\code{CGCTarget}\index{CGCTarget class} class in addition to the
streams defined by the \code{CGTarget}\index{CGTarget class} class.

\begin{indexlist}{CGCTarget}

\begin{ifhtml}
% These streams are not used when writing stars.

\member{CodeStream}{commInit}
This stream is used by the \code{CGCMultiTarget}\index{CGCMultiTarget
class} class (\link{see section}[~\Ref]{CGCMultiTarget}) for the initialization code
associated with send and receive stars.

\member{CodeStream}{wormIn}
This stream is used by the
\code{wormInputCode}\index{wormInputCode!CGCTarget method} method.

\member{CodeStream}{wormOut}
This stream is used by the
\code{wormOutputCode}\index{wormOutputCode!CGCTarget method} method.

\end{ifhtml}

\member{CodeStream}{include}
Include directives are added to this stream by the
\code{addInclude}\index{addInclude!CGCStar method} method of
\code{CGCStar}.

\member{CodeStream}{mainDecls}
Local declarations for variables are added to this stream by the
\code{addDeclaration}\index{addDeclaration!CGCStar method} method of
\code{CGCStar}.

\member{CodeStream}{globalDecls}
Global declarations for variables and functions are added to this
stream by the \code{addGlobal}\index{addGlobal!CGCStar method} method
of \code{CGCStar}.

\member{CodeStream}{mainInit}
Initialization code is added to this stream when the
\code{addCode}\index{addCode!CGStar method} method is invoked from
within the \code{initCode}\index{initCode!Star method} method.

\member{CodeStream}{mainClose}
Code generated when the \code{addCode}\index{addCode!CGStar method}
method is invoked from within the \code{wrapup}\index{wrapup!Star
method} method of stars is placed in this stream.

\end{indexlist}

\section{Buffer Embedding}
\label{CGC Buffer Embedding}

Although many of the methods related to buffer embedding are actually
implemented in the CG domain, only the CGC domain makes use of them at
this time.  The following function is defined as a method of the
\code{CGPortHole}\index{CGPortHole class} class.

\begin{indexlist}{CGPortHole}
\method{void}{embed}{(CGPortHole\& \var{port}, int \var{location} = -1)}
Embed the buffer of \var{port} in the buffer of this porthole with
offset \var{location}.  The default \var{location} of \samp{-1}
indicates that the offset is not yet determined.

\begin{ifhtml}
% These methods are not used when writing a star.

\method{CGPortHole*}{embedded}{()}
Return a pointer to the porthole where this porthole's buffer is embedded.
Return \samp{NULL} if this porthole's buffer is not embedded.

\method{int}{embedding}{()}
Indicate whether or not the buffers of other portholes are embedded in this
porthole's buffer.  Return \samp{TRUE} if this porthole's buffer contains other
buffers, otherwise return \samp{FALSE}.

\method{int}{whereEmbedded}{()}
Return the offset of this porthole's buffer with respect to the buffer in which
it is embedded.

\method{void}{embedHere}{(int \var{location})}
Set the offset of this porthole's buffer with respect to the buffer in which it
is embedded to \var{location}.

\end{ifhtml}

\end{indexlist}

For example, the following statements appear in the \code{setup} method
of the \code{Switch}\indexStarRef{Switch!CGC block} block.  This causes the
buffers of \var{trueOutput} and \var{falseOutput} to be embedded within
the buffer of \var{input}.

\begin{example}
input.embed(trueOutput,0);
input.embed(falseOutput,0);
\end{example}

\section{BDF Stars}
\label{CGC BDF Stars}

Because the class \code{CGCPortHole}\index{CGCPortHole class} is not
derived from \code{BDFPortHole}\index{BDFPortHole class}, the
\code{setBDFParams}\index{setBDFParams!BDFPortHole method} method
described in chapter~\ref{BDF Domain} is not available for code
generation stars.  Use the \code{setRelation} method of
\code{DynDFPortHole} instead.

\begin{indexlist}{DynDFPortHole}

\function{void}{setRelation}{(DFRelation \var{relation}, DynDFPortHole* \var{assoc})}
Specify the \var{relation} of this port with the associated porthole
\var{assoc}.  There are five possible values for \var{relation}:
\begin{description}
\item[DF\_NONE] no relationship.
\item[DF\_TRUE] produces/consumes data only when \var{assoc}
has a \samp{TRUE} particle.
\item[DF\_FALSE] produces/consumes data only when \var{assoc}
has a \samp{FALSE} particle.
\item[DF\_SAME] signal is logically the same as \var{assoc}.
\item[DF\_COMPLEMENT] signal is the logical complement of \var{assoc}.
\end{description}

\end{indexlist}

For example, the following statements describe the relationships among
the portholes of the \code{Switch}\indexStarRef{Switch!CGC block} block.

\begin{example}
trueOutput.setRelation(DF\_TRUE, \&control);
falseOutput.setRelation(DF\_FALSE, \&control);
\end{example}

\section{Tcl/Tk Stars}
\label{CGC Tcl/Tk Stars}

The \code{CGCTclTkTarget} class defines the
\code{tkSetup}\index{tkSetup!CGCTclTkTarget stream for Tcl/Tk stars.
member}.  There is no special code generation function for this stream,
so its name must be used with \code{addCode}\index{addCode!CGStar
method}.  This is usually done from within the
\code{initCode}\index{initCode!CGStar method} method.

\begin{example}
addCode(\var{codeblock}, "tkSetup");
\end{example}

The following functions, which are defined in the file
\file{tkMain.c}\index{tkMain.c}, can be used within codeblocks of
Tcl/Tk stars in the CGC domain.

\begin{sloppypar}
\begin{indexlist}{ function}

\function{void}{errorReport}{(char* \var{message})}
This functions creates a pop-up window containing \var{message}.

\function{void}{makeEntry}{(char* \var{window}, char* \var{name},
char* \var{desc}, char*  \var{initValue}, Tcl\_CmdProc* \var{callback})}
This function creates an entry box in a \var{window}.  The \var{name}
of the entry box must be unique (e.g. derived from the star name).  The
description of the entry box is \var{desc}.  The initial value in the
entry box is \var{initValue}.

A \var{callback} function is called whenever the user enters a
\kbd{RET} in the box.  The argument to the \var{callback} function
will be the value that the user has put in the entry box.  The return
value of the \var{callback} function should be \samp{TCL\_OK}.

\function{void}{makeButton}{(char* \var{window}, char* \var{name},
char* \var{desc}, Tcl\_CmdProc* \var{callback})}
This function creates a push button in a \var{window}.  The \var{name}
of the push button must be unique (e.g. derived from the star name).  The
description of the push button is \var{desc}.

A \var{callback} function is called whenever the user pushes the
button.  The return value of the \var{callback} function should be
\samp{TCL\_OK}.

\function{void}{makeScale}{(char* \var{window}, char* \var{name},
char* \var{desc}, int \var{position}, Tcl\_CmdProc* \var{callback})}
This function creates a scale (with slider) in a \var{window}.  The
\var{name} of the scale must be unique (e.g. derived from the star
name).  The description of the push button is \var{desc}.  The initial
\var{position} of the slider must be between \samp{0} and \samp{100}.

A \var{callback} function is called whenever the user moves the slider
in the scale.  The argument to the \var{callback} function will be the
current position of the slider, which can range from \samp{0} to
\samp{100}.  The return value of the \var{callback} function should be
\samp{TCL\_OK}.

\function{void}{displaySliderValue}{(char* \var{window}, char* \var{name},
char* \var{value})}
This function displays a \var{value} associated with a scale's slider.
The scale is identified by its \var{name} and the \var{window} it is in.
This function must be called by the user of the slider.
Only the first 6 characters of the value will be used.

\end{indexlist}
\end{sloppypar}
