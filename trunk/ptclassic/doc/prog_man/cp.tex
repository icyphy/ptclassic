\c Version $Id$
\node CP Domain
\chapter{CP Domain}

\Author{Seungjun Lee}
\cindex{Lee, S.}
\Contrib{Sangjin Hong \\
Thomas M. Parks}
\cindex{Hong, S.}
\cindex{Parks, T.}

\date{$Date$}

\node Execution Model
\section{Execution Model}

When the simulation starts, the \code{go} method of each block is
converted into a separate process that runs independently.  A process
will \dfn{block} when it tries to communicate with another process and
the communication is not possible.  The process will resume as soon as
the blocking condition is resolved, e.g. the communication channel
becomes available or new data arrives at a porthole.  A process can
explicitly \dfn{suspend} itself for a period of time.  A suspended
process is resumed by the scheduler after the specified amount of time
has passed.  A process \dfn{terminates} when the \code{go} method
returns.

It is assumed that a block executes in zero simulated time:  the global
clock does not advance while a block is running.  A process should
be suspended explicitly to model the execution delay, which allows the
global clock to advance.  Consider a simple server that waits for input
data, processes it, and sends the result back through the output porthole:

\begin{example}
go() \{
    while (TRUE) \{
        msgReceive(data, input);
        result = processData(data);
        waitFor(delay);
        msgSend(result, output);
    \}
\}
\end{example}

The \code{go} method is specified with an infinite loop, which is
typical for a CP block.  In \code{msgReceive}, the process receives
\var{data} from the \var{input} porthole.  The process blocks if
\var{data} is not available and \code{msgReceive} does not return until
it is received.  Once \code{msgReceive} returns, and the server
processes the \var{data} in \code{processData}.  The execution of
\code{processData} is assumed to take zero time and the global clock
does not advance.  To model the execution time, use \code{waitFor}.
The statement \samp{waitFor(\var{delay})} suspends the process and
places it in the scheduler's list of waiting processes to be resumed
after \var{delay} time units.  When the global clock advances by
\var{delay} time units, the server process resumes execution.  It sends
out the \var{result} through the \var{output} porthole.  The execution
blocks until it is sent successfully and \code{msgSend} returns.  This
whole sequence is repeated indefinitely.

\node Writing CP Stars
\section{Writing CP Stars}

A CP star can be viewed as a process that executes independently
once the simulation starts.  Hence, the behavior of a block is usually
described by an infinite loop.  Many methods are provided to make
modeling at a high level of abstraction easier.

Blocks communicate with each other by passing messages through
portholes.  The communication protocol for a porthole is \emph{block on
full/empty} by default.  The following methods of the \code{CPPortHole}
class can be used to change the protocol or examine the state of the
porthole.

\begin{indexlist}{CPPortHole}

\method{void}{setProtocol}{(Protocol \var{mode})}
Set the porthole's protocol to \var{mode}.
Output portholes can have the following protocols:
\begin{description}
\item[BLOCK]  block on full
\item[OVERWRITE] overwrite on full
\item[IGNORE] ignore on full
\end{description}
Input portholes can have the following protocols:
\begin{description}
\item[BLOCK] block on empty
\item[PREVIOUS] previous on empty
\item[IGNORE] ignore on empty
\end{description}

\method{Protocol}{getProtocol}{()}
Return the porthole's protocol.

\method{int}{transReady}{()}
Return \samp{TRUE} if the porthole is ready for a communication transaction.

\end{indexlist}

The following methods of the \code{CPStar} class are used for inter-process
communication and process synchronization.

\begin{indexlist}{CPStar}

\method{double}{currentTime}{()}
Get the current value of the global clock, which is maintained by the
scheduler.

\method{double}{getRandom}{(double \var{max})}
Return a random number between zero and \var{max}.  This is useful for
modeling stochastic behavior.

\method{void}{waitFor}{(double \var{delay})}
Suspend the current process until \var{delay} units of simulated time
have elapsed.

\method{int}{msgReceive}{(int& \var{data}, InCPPort& \var{port}, double \var{timeout} = -1)}
\code{(float& \var{data}, InCPPort& \var{port}, double \var{timeout} = -1)} \\
\code{(Envelope& \var{data}, InCPPort& \var{port}, double \var{timeout} = -1)} \\
Receive \var{data} through a \var{port}.  Multiple versions are
available to support different types of \var{data}.  Return immediately
if the protocol is either \samp{PREVIOUS} or \samp{IGNORE}.  If the
protocol is \samp{BLOCK}, then return only after the transaction is
complete or \var{timeout} units of simulated time have elapsed.  The
default \var{timeout} of \samp{-1} corresponds to an infinite amount of
time.

\method{int}{msgReceive}{(MultiInCPPort& \var{ports})}
Clear out received data from a set of input \var{ports}.

\method{int}{msgSend}{(int \var{data}, OutCPPort& \var{port}, double \var{timeout} = -1)}
\code{(float \var{data}, OutCPPort& \var{port}, double \var{timeout} = -1)} \\
\code{(Envelope \var{data}, OutCPPort& \var{port}, double \var{timeout} = -1)} \\
Send \var{data} through an output \var{port}.  Multiple versions are
available to support different types of \var{data}.  Return immediately
if the protocol is either \samp{OVERWRITE} or \samp{IGNORE}.  If the
protocol is \samp{BLOCK}, then return only after the transaction is
complete or \var{timeout} units of simulated time have elapsed.  The
default \var{timeout} of \samp{-1} corresponds to an infinite amount of
time.

\method{int}{msgSend}{(int \var{data}, MultiOutCPPort& \var{ports})}
\code{(float \var{data}, MultiOutCPPort& \var{ports})} \\
\code{(Envelope \var{data}, MultiOutCPPort& \var{ports})} \\
Broadcast \var{data} through a set of output \var{ports}.  Multiple
versions are available to support different types of \var{data}.  Return
immediately if the protocol is either \samp{OVERWRITE} or
\samp{IGNORE}.  If the protocol is \samp{BLOCK}, then return only after
the transaction is complete.

\method{int}{waitAll}{(int \var{number}, ...)}
Wait for all of the listed portholes to be ready for communication.
The first argument specifies the \var{number} of remaining arguments,
which must be of type \samp{PortHole*} or \samp{MultiPortHole*}.
Return the number of ready portholes, or \samp{-1} on error.

\method{int}{TWaitAll}{(double \var{timeout}, int \var{number}, ...)}
This method is the same as \code{waitAll} except that it will return
after \var{timeout} units of simulated time have elapsed even if the
listed portholes are not ready.

\method{int}{waitAny}{(int \var{number}, ...)}
Wait for any of the listed portholes to be ready for communication.
The first argument specifies the \var{number} of remaining arguments,
which must be of type \samp{PortHole*} or \samp{MultiPortHole*}.
Return the number of ready portholes, or \samp{-1} on error.

\method{int}{TWaitAny}{(double \var{timeout}, int \var{number}, ...)}
This method is the same as \code{waitAny} except that it will return
after \var{timeout} units of simulated time have elapsed even if the
listed portholes are not ready.

\method{CPPortHole*}{waitOne}{(int \var{number}, ...)}
Wait for one of the listed portholes to be ready for communication.
The first argument specifies the \var{number} of remaining arguments,
which must be of type \samp{PortHole*} or \samp{MultiPortHole*}.
Return a pointer to a ready porthole.  If more than one porthole is
ready, then choose a porthole randomly.

\method{CPPortHole*}{TWaitOne}{(double \var{timeout}, int \var{number}, ...)}
This method is the same as \code{waitAny} except that it will return
after \var{timeout} units of simulated time have elapsed even if the
listed portholes are not ready.  In this case the return value will be
\samp{NULL}.

\end{indexlist}

The methods \code{waitAll}, \code{waitAny}, and \code{waitOne} only
examine the availability of data; they do not actually
transfer data.  These methods should be used with \code{msgReceive} or
\code{msgSend} to complete the data transaction.  The \code{waitAll}
method can be used to describe \emph{data-driven} behavior as in SDF.
The following example shows the usage of \code{waitAll} in the description
of two-input adder:

\begin{example}
go \{
    while (TRUE) \{
        waitAll(2, &in1, &in2);
        msgReceive(x1, in1);
        msgReceive(x2, in2);
        waitFor(delay);
        msgSend(x1+x2, out);
    \}
\}
\end{example}

The \code{waitAny} method is useful for \emph{event-driven} modeling as in the
DE domain.  It is usually followed by \code{transReady} to
determine which portholes are ready.

\begin{example}
go \{
    while (TRUE)\{
        waitAny(2, &in1, &in2);
        if (transReady(in1)) msgReceive(data, in1);
        if (transReady(in2)) msgReceive(data, in2);
        msgSend(data, out);
    \}
\}
\end{example}

The \code{waitOne} method can describe non-deterministic behavior.  It
is implemented as the combination of \code{waitAny} and random
selection.  It waits for any of the portholes to become ready.  When
there are multiple ready portholes at the same time, it chooses one
randomly.

The following code shows the usage of \emph{TWaitAny()} in clock generator with 
asynchronous \var{reset}.  It generates pulses with a given period.  After
generating a pulse, it waits for a certain interval unless there is a
\var{reset} signal.  If the \var{reset} arrives during the waiting period,
the clock restarts from the moment the \var{reset} signal is asserted.  

\begin{example}
go \{
    while (TRUE)\{
        msgSend(data, output);
        if (TWaitAny(delay, 1, &reset)) msgReceive(data, reset);
    \}
\}
\end{example}

\subsection{Support for ANYTYPE}

Ptolemy allows the data type of a porthole to be determined
dynamically.  If a porthole's data type is specified as
\samp{ANYTYPE}, then the data type is inherited from the porthole it is
connected to.  This scheme is quite useful for implementing generic
blocks such as \code{Fork} \code{Queue} and \code{Merge}.  Within these
blocks, data is never evaluated but is just transmitted untouched.

\samp{ANYTYPE} is also supported in the CP domain, but a little caution
needs to be taken.  A CP block sends and receives data through
\code{msgReceive} and \code{msgSend}, and the data type must be
specified when they are called.  Therefore, these methods cannot be called for
portholes with \samp{ANYTYPE}.  In this case, the \code{receiveData} and
\code{sendData} methods of the \code{CPPortHole} class must be
used directly.  For example:

\begin{example}
input \{
    name \{ in \}
    type \{ anytype \}
\}
output \{
    name \{ out \}
    type \{ =in \}
\}
protected \{
    Queue buffer;
\}
go() \{
    in.receiveData();
    buffer.put((in%0).clone());
    Particle *p = (Particle*) buffer.get();
    out%0 = *p;
    out.sendData();
\}
\end{example}
