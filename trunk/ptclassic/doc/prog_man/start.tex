\c Version $Id$
\node Getting Started
\chapter{Getting Started}
\Author{T. M. Parks}
\date{$Date$}
If you want to extend Ptolemy by modfying or
adding a new scheduler, target, or
even an entire domain, it is recommended that you create a duplicate
directory hierarchy.  This allows you to experiment with and fully test
any changes before incorporating them into the ``official'' version of
Ptolemy.  It also means that all of the existing makefiles will work
without modification because all of the paths specified are relative to
the root of the hierarchy.

The most direct way to do this is to copy the entire
Ptolemy hierarchy.  This could be done with a command such as:
\begin{example}
cp -r ~ptolemy ~/ptolemy
\end{example}
which would create a copy of the hierarchy in your home directory.
Because this method requires excessive disk space, many developers
prefer to use symbolic links when creating a duplicate hierarchy.

\node Aliases for Managing Symbolic Links
\section{Aliases for Managing Symbolic Links}

Below are several \code{csh} aliases that can be helpful when managing a
duplicate hierarchy that is implemented with symbolic links:
\begin{example}
alias obj 'echo $cwd | sed s:/src:/obj.$\{ARCH\}:'
alias src 'echo $cwd | sed s:/obj.$\{ARCH\}:/src:'
alias pt  'echo $cwd | sed s:$\{HOME\}/Ptolemy:$\{PTOLEMY\}:'
alias ptl 'ln -s `pt`/* .'
alias sw  'mv \back !^ swap$$; mv .\back !^ \back !^; mv swap$$ .\back !^'
alias exp 'mkdir .\back !^; sw \back !^; cd \back !^; ptl'
alias rml '\back rm -f `\back ls -F \back !* | sed -n s/@\back $//p`'
alias mkl 'rml make*; ln -s `vpath`/make* .'
\end{example}

\node The obj and src Aliases
\subsection{The \protect\code{obj} and \protect\code{src} Aliases}

The \code{obj} alias gives the name of the object-file directory
corresponding to the current source-file directory.  The \code{src}
alias does the opposite of this.  These aliases make it easy to switch
back and forth between directories during the edit-compile-debug
development cycle, as shown in the following sequence of
commands:
\begin{example}
vi SDFStar.h
cd `obj`
make
cd `src`
vi SDFStar.h
\end{example}

\node The pt Alias
\subsection{The \protect\code{pt} Alias}

The \code{pt} alias gives the name of the ``official''
Ptolemy directory that corresponds to the current directory.  This assumes
that you have the environment variable \code{$PTOLEMY} set to the root
directory of
the ``official'' version of Ptolemy,
and that your
private version is in \file{~/Ptolemy}.  If this is not the case, then
you should make suitable modifications to the \code{pt} alias.  This
alias is useful when you want to make a symbolic link to or otherwise
access the ``official'' version of a file:
\begin{example}
ln -s `pt`/SCCS .
\end{example}

\node The ptl Alias
\subsection{The \protect\code{ptl} Alias}

The \code{ptl} alias uses the \code{pt} alias to create, in the current
directory, symbolic links to all the files in the corresponding
``official'' directory.  This is useful for quickly filling in the
branches of a new directory in your private hierarchy.
\begin{example}
mkdir .stars
sw stars
cd stars
ptl
\end{example}

\node The sw Alias
\subsection{The \protect\code{sw} Alias}
The \code{sw} alias, which was used above, swaps a file \var{filename}
with another file \var{.filename}.  The period at the beginning of the
second file's name makes it invisible unless you use the \samp{-a}
option of the \code{ls} command.  If you use the \samp{-F} option of
the \code{ls} command, then it is easy to see which files in a
directory are symbolic links.

\node The exp Alias
\subsection{The \protect\code{exp} Alias}
The \code{exp} aliases combines the functions of the \code{ptl} and \code{sw}
aliases into one, making the common task of expanding a branch in the
directory hierarchy easy.  It invokes the same set of commands shown in the
example for the \code{ptl} alias (\pxref{The ptl Alias}),
leaving you in the new
directory ready to issue another \code{exp} command.
\begin{example}
cd ~/Ptolemy
exp src
exp domains
exp sdf
exp dsp
exp stars
\end{example}

\node The rml Alias
\subsection{The \protect\code{rml} Alias}
The \code{rml} alias removes symbolic links in the current directory.
Without an argument, it removes all the visible symbolic links.  Any
arguments are passed on to the \code{ls} command.  So, to remove
\emph{all} symbolic links, including those that are invisible, use the
\samp{-a} option:
\begin{example}
rml -a
\end{example}
You can also give file names as arguments to remove just some of the symbolic
links:
\begin{example}
rml *.o
\end{example}

\node Creating a Duplicate Hierarchy
\section{Creating a Duplicate Hierarchy}

Let's look at an example to see how these aliases can be used.  Suppose
you want to modify an existing file that is part of the kernel for the
SDF domain.  You will need a private copy of the file that is
writable.  This allows you to make your changes without affecting the
``official'' version of Ptolemy.  In order to test your change, you
will have to build a private version of the interpreter \code{ptcl} or
the graphical interface \code{pigiRpc}.

\noindent
First, create the root directory for your duplicate hierarchy.
\begin{example}
mkdir ~/Ptolemy
\end{example}
Then go into that directory and create symbolic links to all files in the
corresponding ``official'' Ptolemy directory.
\begin{example}
cd ~/Ptolemy
ptl
\end{example}
You will want to have a private version of the
\file{lib.$ARCH} directory
so that you won't modify the ``official'' version of any library or
object files.
\begin{example}
cd ~/Ptolemy
exp lib.$ARCH
\end{example}
You will also want a private
\file{obj.$ARCH} directory
for the same
reason.  In this example, the tree is expanded down to the \file{sdf}
directory.
\begin{example}
cd ~/Ptolemy
exp obj.$ARCH
exp domains
exp sdf
\end{example}
If you are modifying code in the \file{sdf/kernel} directory, then you
will want to expand it as well.  Once expanded, you will want remove
the \file{make.template} and \file{makefile} links (which point to the
``official'' Ptolemy files) and replace them with links that use relative
paths to refer to your private versions of these files (in case you make
changes to them).
\begin{example}
exp kernel
mkl
\end{example}
If you make changes in the \file{sdf/kernel} directory, then there is a good
chance that object files in \file{sdf/dsp} and other directories
will also have to be recompiled.  Thus, you will want to expand
these directories (and any subdirectories below them) as well.  Remember to
replace the \file{make.template} and \file{makefile} links as in the
\file{sdf/kernel} directory.
\begin{example}
exp dsp
mkl
exp stars
mkl
\end{example}
Because of the way symbolic links work, it is important to remove the
links for the \file{.o} and \file{.a} files in the directories you have
just created.  You can do this by issuing a \samp{make realclean}
command in the \file{obj.$ARCH/domains/sdf} directory.  This will
recursively clean out all the subdirectories.  You could also do this
manually by issuing a \samp{rml *.o *.a} command in each directory.

You will also need a private version of the \file{src} directory.
\begin{example}
cd ~/Ptolemy
exp src
exp domains
exp sdf
exp kernel
\end{example}

At any point after this, it is possible to switch back and forth
between private and
``official'' versions of these directories with the \code{sw} alias.
In fact, you just used it (as part of the \code{exp} alias) to
switch to the private versions of the
\file{obj.$ARCH}, \file{lib.$ARCH}, and \file{src} directories.

