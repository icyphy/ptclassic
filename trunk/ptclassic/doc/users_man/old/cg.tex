\c Version $Id$

\node CG Domain
\chapter{CG Domain}

\Author{ Joseph Buck \\
Soonhoi Ha \\
Edward A. Lee \\
Praveen K. Murthy \\
Thomas M. Parks \\
Jos\'{e} Luis Pino \\
Kennard White}
\date{$Date$}

The CG domain and those derived from it generate code from a system
description instead of interpreting it in a simulation.
Different code
generation domains produce code for different computer languages, such
as \code{C} (\pxref{CGC Domain}),
\cindex{code generation in C}
\cindex{C code generation}
\cindex{CGC domain}
\code{Silage} (\pxref{Silage Domain}),
\cindex{code generation in Silage}
\cindex{Silage domain}
\cindex{Silage code generation}
and assembly language(\pxref{CG56 Domain}).
\cindex{code generation in assembly}
\cindex{assembly code generation}
\cindex{CG56 domain}
Most code
generation domains support only the SDF model of computation
(\pxref{SDF Domain}),
\cindex{synchronous dataflow}
\cindex{dataflow, synchronous}
but some also support the BDF model
(\pxref{BDF Domain}).
\cindex{Boolean-controlled dataflow}
\cindex{dataflow, Boolean-controlled}
The CG domain itself does not generate real code:  it is intended
for demonstration and testing purposes only.  You can view the
generated code and display a Gantt chart for parallel
schedules.  In other code generation domains real code is generated,
compiled, down loaded, and executed, all under the control of the
selected target.

\node CG Single-Processor Targets
\section{Single-Processor Targets}

There are two single-processor targets available in the CG domain:
\code{default-CG} and \code{bdf-CG}.
\tpindex{default-CG target}
\tpindex{bdf-CG target}
The \code{default-CG} target makes only the \var{directory} and
\var{Looping Level} parameters available to the user.
\vrindex{directory parameter}
\vrindex{Looping Level parameter}
It uses a SDF scheduling algorithm for a single processor.
\cindex{synchronous dataflow}
\cindex{dataflow, synchronous}
\xref{SDF Domain}, for more details on scheduling.
The \code{bdf-CG} target makes only the \var{directory} parameter
available to the user.
\vrindex{directory parameter}
It uses a BDF
scheduling algorithm for a single processor.
\xref{BDF Domain},
\cindex{Boolean-controlled dataflow}
\cindex{dataflow, Boolean-controlled}
for more details on this scheduling algorithm.

All code is written to and compiled and run on the computer specified
by the \var{host} parameter.
\vrindex{host parameter}
If a remote
computer is specified by \var{host} then \code{rsh} commands are used
to place files on that computer and to invoke the compiler.  You should
verify that your \file{.rhosts} file is properly configured so that
\code{rsh} will work.  The code is placed in the directory specified by
the \var{directory}
\vrindex{directory parameter}
parameter, in files
named according to the \var{file}
\vrindex{file parameter}
parameter.

The parameters \var{display?}, \var{compile?}, \var{load?}, and \var{run?}
\vrindex{display? parameter}
\vrindex{compile? parameter}
\vrindex{load? parameter}
\vrindex{run? parameter}
are provided for debugging and development.  For example, during
development you may want to set \var{display?} to \samp{TRUE} and set
\var{compile?} to \samp{FALSE}.  This would allow you to view the
generated code without compiling it.  Later you may want to set
\var{compile?} back to \samp{TRUE} but set \var{run?} to \samp{FALSE}.
This would allow you to verify that the generated code compiles
correctly without attempting to run the program, or you could start the
program manually under the control of a debugger.  If compilation
fails, then the program will not be run regardless of the state of
\var{run?}.  Once everything works, you may want to set \var{display?}
to \samp{FALSE} to avoid the annoyance of having the code displayed for
every run.

The complete list of parameters available to all code generation
targets is given below.  Some targets hide these parameters while
others make them available to the user.

\begin{statelist}
\state{STRING}{host}{}
\vrindex{host parameter}
The name of the computer where programs will be compiled and run.  If
the name is blank, which is the default, then programs will be compiled
and run on the same computer where Ptolemy is running.

\state{STRING}{directory}{$HOME/PTOLEMY_SYSTEMS}
\vrindex{directory parameter}
The destination directory for files created in the code generation
process.  If the directory does not already exist, it will be created.
All files are written into this directory on the \var{host} computer.

\state{STRING}{file}{}
\vrindex{file parameter}
The base name for files created in the code generation process.  If the
name is blank, which is the default, then a suitable name will be
generated automatically.  File name extensions, such as \file{.c} and
\file{.asm}, are automatically added to this name to form the complete
file name and should not be specified as part of \var{file}.

\state{INT}{display?}{TRUE}
\vrindex{display? parameter}
Controls the display of the generated code.

\state{INT}{compile?}{TRUE}
\vrindex{compile? parameter}
Controls the invokation of the compiler after code generation.

\state{INT}{load?}{TRUE}
\vrindex{load? parameter}
Controls loading of the program onto the \var{host} computer after
compilation.  Some targets do not implement separate load and run
operations and thus do not make \var{load?} available to the user.

\state{INT}{run?}{TRUE}
\vrindex{run? parameter}
Controls the running of the program after compilation.

\state{INT}{Looping Level}{0}
\vrindex{Looping Level parameter}
Determines which scheduling algorithm is used.  If set to \samp{0}, an
in-line schedule will be generated.  If set to \samp{1}, then a
conservative loop-generating algorithm is used.  If set to \samp{2},
then a more exhaustive loop-generating algorithm is used.  The
loop-generating algorithms generally produce more compact code but may
take longer to run.

\end{statelist}

\node CG Multi-Processor Targets
\section{Multi-Processor Targets}

There are two multi-processor targets included in the CG domain:
\code{FullyConnected} and \code{SharedBus}.
\tpindex{FullyConnected target}
\tpindex{SharedBus target}
Both use Gil Sih's
\cindex{Sih, Gil}
multi-processor scheduling algorithms, but specifiy different
communication topologies among the processors.  Both have the same
parameters.

The \var{Looping Level} parameter is not used by multi-processor
\vindex{Looping Level}
targets, so it is not available to the user.  All other parameters
described in \nxref{CG Single Processor Targets}, are defined for
multi-processor targets.  In addition, the following paramters are
defined for all multi-processor targets.  Some targets may also define
additional parameters.

\begin{statelist}
\state{INT}{nprocs}{2}
\vindex{nprocs parameter}
The number of processors.

\state{INT}{inheritProcessors}{NO}
\vindex{inheritProcessors parameter}

\state{INT}{sendTime}{1}
\vindex{sendTime parameter}
The time required to transfer one datum.

\state{INT}{oneStarOneProc}{NO}
\vindex{oneStarOneProc parameter}
Controls whether or not \emph{all} firings of a star will be mapped to
a single processor.

\state{INT}{manualAssignment}{NO}
\vindex{manualAssignment parameter}
Allows the user to manually specify the assignment of stars to processors.

\state{INT}{adjustSchedule}{NO}
\vindex{adjustSchedule parameter}
Allows the user to override the previously obtained schedule by manual
assignment.

\state{STRINGARRAY}{childType}{default-CG}
\vindex{childType parameter}
The target type for child processors.

\state{STRINGARRAY}{resources}{}
\vindex{resources parameter}
Resourses of the child targets.  Use a \samp{;} to separate the resources of
the different targets.

\state{INTARRAY}{relTimeScales}{1}
\vindex{relTimeScales parameter}
The relative time scales of the child targets.

\state{INT}{ganttChart}{YES}
\vindex{ganttChart parameter}
Display a Gantt chart for the parallel schedule.

\state{STRING}{logFile}{}
\vindex{logFile parameter}
Name of log file used by the scheduler.  If left empty, which is the default,
then no log file will be created.

\state{INT}{amortizedComm}{NO}
\vindex{amortizedComm parameter}

\state{INT}{ignoreIPC}{NO}
\vindex{ignoreIPC parameter}
Force inter-processor communication costs to be ignored when computing the
schedule.

\state{INT}{overlapComm}{NO}
\vindex{overalpComm parameter}
Communication and computation can be overlapped.

\state{INT}{useCluster}{NO}
\vindex{useCluster parameter}
Use Gil Sih's declustering scheduling algorithm.

\end{statelist}

\node CG Star Overview
\section{An Overview of CG Stars}

\begin{figure}
\begin{center}
\ \psfig{figure=/users/ptdesign/src/domains/cg/icons/cg.pal.ps}
\end{center}
\caption{Palette of stars available in the CG domain.}
\label{cg-stars}
\end{figure}

Figure \ref{cg-stars} shows the palette of stars available in the
CG domain.  Each star is described briefly below.  Please refer to the
Almagest Star Atlas for a more detailed description.

\begin{description}
\item[Source]
Generic code generator source star.  It produces \var{n} samples on
\var{output} for each firing.

\item[Sink]
Swallows input samples.  It consumes \var{n} samples on \var{input} for
each firing.

\item[Through]
Generates conditional code, depending on the value of the Boolean state
\var{control}.  Samples are passed unconditionally from \var{input} to
\var{output}.

\item[MultiIn]
Consumes one sample from each \var{input} and produces one sample on
\var{output}.

\item[MultiOut]
Consumes one sample from \var{input} and produces one sample on each
\var{output}.

\item[RateChange]
Consumes \var{consume} samples from \var{input} and produces
\var{produce} samples on \var{output}.

\end{description}

\node CG Demo Overview
\section{An Overview of CG Demos}

\begin{figure}
\begin{center}
\ \psfig{figure=/users/ptdesign/src/domains/cg/demo/init.pal.ps}
\end{center}
\caption{Palette of demos available in the CG domain.}
\label{cg-demos}
\end{figure}

Figure \ref{cg-demos} shows the palette of demos available in the
CG domain.  Each demo is described briefly below.  Please refer to the
Almagest Star Atlas for a more detailed description.

\begin{description}
\item[useless]
This is an extremely simple demo which demonstrates
the most basic code generation capabilities.

\item[schedTest]
This is a simple multi-processor code generation demo.  By editing the
parameters of the \code{RateChange} star, you can make the demo a bit more
interesting.

\item[Sih-4-1]
This demo illustrates the scheduling problem of Figure 4-1 from Gil Sih's
dissertation.

\item[pipeline]
This demo shows how retiming can be used to increase the amount of
exploitable parallelism through pipelined scheduling.  The \var{retime}
parameter can be used to turn reiming on and off so that you can see
the schedule with and without it.

\end{description}
