\c Version $Id$

\node CG Domain
\chapter{CG Domain}

\Author{ J. Buck \\
Soonhoi Ha \\
Edward A. Lee \\
Thomas M. Parks \\
Jos\'{e} Pino \\
Kennard White}
\date{$Date$}

\node CG Single-Processor Targets
\section{Single-Processor Targets}

All code is written to and compiled and run on the computer specified
by the \var{host} parameter.  \vrindex{host parameter} If a remote
computer is specified by \var{host} then \code{rsh} commands are used
to place files on that compuer and to invoke the compiler.  You should
verify that your \file{.rhosts} file is properly configured so that
\code{rsh} will work.  The code is placed in the directory specified by
the \var{directory} \vrindex{directory parameter} parameter, in files
named according to the \var{file} \vrindex{file parameter} parameter.

The parameters \var{display?}, \var{compile?}, \var{load?},
and \var{run?} are
\vrindex{display? parameter}
\vrindex{compile? parameter}
\vrindex{load? parameter}
\vrindex{run? parameter}
provided for debugging and development.  For example, during
development you may want to set \var{display?} to \samp{TRUE} and set
\var{compile?} to \samp{FALSE}.  This would allow you to view the
generated code without compiling it.  Later you may want to set
\var{compile?} back to \samp{TRUE} but set \var{run?} to \samp{FALSE}.
This would allow you to verify that the generated code compiles
correctly without attempting to run the program.  (If compilation
fails, then the program will not be run regardless of the state of
\var{run?}.)  Once everything works, you may want to set \var{display?}
to \samp{FALSE} to avoid the annoyance of having the code displayed for
every run.

The complete list of parameters available to all code generation
targets is given below.  Some targets hide these parameters while
others make them available to the user.

\begin{statelist}
\state{STRING}{host}{}
\vrindex{host parameter}
The name of the computer where programs will be compiled and run.  If
the name is blank, which is the default, then programs will be compiled
and run on the same computer where Ptolemy is running.

\state{STRING}{directory}{$HOME/PTOLEMY_SYSTEMS}
\vrindex{directory parameter}
The destination directory for files created in the code
generation process.  If the directory does not already exist,
it will be created.  All files are written into this directory on the
\var{host} computer.

\state{STRING}{file}{}
\vrindex{file parameter}
The base name for files created in the code generation process.
If the name is blank, which is the default, then a suitable name will be
generated automatically.
The \file{.c} extension is automatically added to this name to form the
complete file name.

\state{INT}{display?}{TRUE}
\vrindex{display? parameter}
Controls the display of the generated code.

\state{INT}{compile?}{TRUE}
\vrindex{compile? parameter}
Controls the invokation of the compiler after code generation.

\state{INT}{load?}{TRUE}
\vrindex{load? parameter}
Controls downloading of the program to the \var{host} computer after
compilation.  Some targets do not implement separate load and run
operations and thus do not make \var{load?} available to the user.

\state{INT}{run?}{TRUE}
\vrindex{run? parameter}
Controls the running of the program after compilation.

\state{INT}{Looping Level}{0}
\vrindex{Looping Level parameter}
Determines which scheduling algorithm is used.  If set to \samp{0}, an
in-line schedule will be generated.  If set to \samp{1}, then a
conservative loop-generating algorithm is used.  If set to \samp{2},
then a more exhaustive loop-generating algorithm is used.  The
loop-generating algorithms generally produce more compact code but may
take longer to run.
\end{statelist}

\node default-CG
\subsection{\protect\code{default-CG}}

The default-CG target makes only the \var{directory} and \var{Looping Level}
parameters available to the user.
\vrindex{directory parameter}
\vrindex{Looping Level parameter}

\node bdf-CG
\subsection{\protect\code{bdf-CG}}

\node CG Multi-Processor Targets
\section{Multi-Processor Targets}

\begin{statelist}
\state{INT}{nprocs}{}
\state{INT}{inheritProcessors}{}
\state{INT}{sendTime}{}
\state{INT}{oneStarOneProc}{}
\state{INT}{manualAssignment}{}
\state{INT}{adjustSchedule}{}
\state{STRINGARRAY}{childType}{}
\state{STRINGARRAY}{resources}{}
\state{INT}{relTimeScales}{}
\state{INT}{ganttChart}{}
\state{STRING}{logFile}{}
\state{INT}{amortizedComm}{}
\state{INT}{ignoreIPC}{}
\state{INT}{overlapComm}{}
\state{INT}{useCluster}{}
\end{statelist}

\node SharedBus
\subsection{\protect\code{SharedBus}}

\node FullyConnected
\subsection{\protect\code{FullyConnected}}

\node CG Star Overview
\section{An Overview of CG Stars}

\node CG Demo Overview
\section{An Overview of CG Demos}
