\c Version $Id$

\c Path to the CGC domain.
\newcommand{\CGC}{\PTOLEMY/src/domains/cgc}

\node CGC Domain
\chapter{CGC Domain}

\Author{Joseph Buck \\
Soonhoi Ha \\
Edward A. Lee \\
Thomas M. Parks}
\date{$Date$}

The \dfn{CGC domain} of Ptolemy generates code
for the \code{C} programming language.
\cindex{code generation in C}
\cindex{C code generation}
\cindex{CGC domain}
This domain supports both synchronous dataflow
(SDF, \pxref{SDF Domain})
\cindex{synchronous dataflow}
\cindex{dataflow, synchronous}
and Boolean-controlled dataflow
(BDF, \pxref{BDF Domain})
\cindex{Boolean-controlled dataflow}
\cindex{dataflow, Boolean-controlled}
models of computation.  The model 
associated with a particular program graph is determined by
which target is selected.  The \code{bdf-CGC} target supports the BDF
model, while all other targets in the CGC domain support only the
SDF model.  Code can be generated for both single-processor
and multi-processor computers.  The targets that support
single processors include \code{default-CGC}, \code{TclTk_Target}, and
\code{bdf-CGC}.  The multi-processor target is \code{unixMulti_C}.

\node CGC Targets
\section{CGC Targets}

The targets of the CGC domain
generate \code{C} code from dataflow program graphs.
Code generation is controlled by the \var{host}, \var{directory}, and
\var{file} parameters as described in \nxref{CG Targets}.
\vindex{host parameter}
\vindex{directory parameter}
\vindex{file parameter}
The command used to compile the code is determined by the
\var{compileCommand}, \var{compileOptions}, and \var{linkOptions}
\vindex{compileCommand parameter}
\vindex{compileOptions parameter}
\vindex{linkOptions parameter}
parameters.  Compilation and execution are controlled by the
\var{display?}, \var{compile?}, and \var{run?}
\vindex{display? parameter}
\vindex{compile? parameter}
\vindex{run? parameter}
parameters, also described in \nxref{CG Targets}.
The other parameters common to all CGC targets are listed below.  Not
all of these parameters are made available to the user by every
target, and some targets define additional parameters.

\begin{statelist}{parameter}
\state{INT}{staticBuffering}{TRUE}
If \samp{TRUE}, then attempt to use static, compile-time addressing of data
buffers between stars.  Otherwise, use dynamic, run-time addressing.

\state{STRING}{funcName}{main}
The name of the main function.  The default value of \samp{main} is suitable
for generating stand-alone programs.  Choose another name if you wish to use
the generated code as a procedure that is called from your own main program.

\state{STRING}{compileCommand}{cc}
Command name of the compiler.

\state{STRING}{compileOptions}{}
Options passed to the compiler.

\state{STRING}{linkOptions}{-lm}
Options passed to the linker.

\state{STRINGARRAY}{resources}{STDIO}
List of abstract resources that the host computer has.
\end{statelist}

\node CGC Single-Processor Targets
\subsection{Single-Processor Targets}

\begin{table}
\centering
\c Align the text so that it will look right in the info file.
\begin{tabular}{lll}
    \var{compile?}      &\var{file}       &\var{Looping Level}\\
    \var{compileCommand}&\var{funcName}   &\var{resources}\\
    \var{compileOptions}&\var{host}       &\var{run?}\\
    \var{directory}     &\var{linkOptions}&\var{staticBuffering}\\
    \var{display?}      &                 &\\
\end{tabular}
\caption{Parameters for the \protect\code{default-CGC} target.}
\label{default-CGC-parameters}
\end{table}

The \code{default-CGC} target
\tindex{default-CGC target}
generates \code{C} code for a single processor from a SDF program
graph.  The parameters available to the user are shown in
table\tie\ref{default-CGC-parameters}.  \xref{CG Targets} , and
\nxref{CGC Targets} for detailed descriptions of these parameters.

The \code{TclTk_Target} target,
\tindex{TclTk_Target target}
which is derived from the \code{default-CGC} target,
must be used when Tcl/Tk stars are present in
the program graph.  The parameters differ from those of the
\code{default-CGC} target only in their default values.

\begin{statelist}{parameter}

\state{INT}{Looping Level}{1}

\state{STRING}{funcName}{go}

\state{STRING}{compileOptions}{-I$PTOLEMY/tcltk/tk/include \\
-I$PTOLEMY/tcltk/tcl/include \\
-I$PTOLEMY/src/domains/cgc/tcltk/lib}

\state{STRING}{linkOptions}{-L$PTOLEMY/tcltk/tk.$ARCH/lib \\
-L$PTOLEMY/tcltk/tcl.$ARCH/lib \\
-L$PTOLEMY/lib.$ARCH \\
-L$PTOLEMY/tcltk/tk.$ARCH/lib \\
-L/usr/X11/lib -ltk -ltcl -lptk -lXpm -lX11 -lm}

\end{statelist}

The \code{bdf-CGC} target
\tindex{bdf-CGC target}
supports the BDF model of
computation.  It must be used when BDF stars are present in the program
graph.  It can also be used with program graphs that contain only
SDF stars.
The \code{bdf-CGC} target has the same parameters as the
\code{default-CGC} target with the exception that
the \var{Looping Level} parameter is absent.  This is because a
loop-generating algorithm is always used for scheduling.
\xref{BDF Domain} for details.

\node CGC Multi-Processor Targets
\subsection{Multi-Processor Targets}

\begin{table}
\centering
\c Align the text so it will look right in the info file.
\begin{tabular}{lll}
    \var{adjustSchedule}&\var{ignoreIPC}        &\var{overlapComm}\\
    \var{amortizedComm} &\var{inheritProcessors}&\var{portNumber}\\
    \var{childType}     &\var{logFile}          &\var{relTimeScales}\\
    \var{compile?}      &\var{machineNames}     &\var{resources}\\
    \var{directory}     &\var{manualAssignment} &\var{run?}\\
    \var{display?}      &\var{nameSuffix}       &\var{sendTime}\\
    \var{file}          &\var{nprocs}           &\var{userCluster}\\
    \var{ganttChart}    &\var{oneStarOneProc}   &\\
\end{tabular}
\caption{Parameters for the \protect\code{unixMulti_C} target.}
\label{unixMultiC-parameters}
\end{table}

The multi-processor target supported in the CGC domain is \code{unixMulti_C}.
It generates code for multiple networked workstations, using a shared bus
configuration for scheduling purposes.  In addition to the target
parameters described in
\nxref{CGC Targets}, and \nxref{CG Targets},
this target defines the user parameters listed below.
Table\tie\ref{unixMultiC-parameters} gives the complete list of parameters for
the \code{unixMulti_C} target.

\begin{statelist}{parameter}
\state{INT}{portNumber}{7654}
The starting port number used by send/receive stars.  The port number
is incremented for each send/receive pair.  It is the
responsibility of the user to ensure that the port number does not
conflict with any that may already be in use.

\state{STRING}{machineNames}{herschel}
The names of the networked workstations which form the
multi-processor.  The names should be separated by a comma (\samp{,}).

\state{STRING}{nameSuffix}{}
The domain suffix for the workstations named in \var{machineNames}.  If left
blank, which is the default, then the workstations are assumed to be part of
the local domain.  Otherwise, specify the proper domain name, including
a leading period, such as \samp{.berkeley.edu}.
\end{statelist}

\node CGC Star Overview
\section{An Overview of CGC Stars}

\begin{figure}
\centering
\ \psfig{figure=\CGC/icons/main.pal.ps}
\caption{Top-level palette of stars in the CGC domain.}
\label{figure CGC stars}
\end{figure}

Figure\tie\ref{figure CGC stars} shows the top-level palette of CGC
stars.  The stars are divided into categories:  sources, sinks,
aritmetic functions, nonlinear functions, control, conversion, signal
processing, boolean-controlled dataflow, and Tcl/Tk stars.  Many of the
stars in the CGC domain have equivalent counterparts in the SDF
domain.  \xref{SDF Star Overview} for brief descriptions of these
stars.  Brief descriptions of the stars unique to the CGC domain are
given in the following sections.  Refer to the Star Atlas for more
detailed descriptions.

\node CGC Source Stars
\subsection{Source Stars}

\begin{figure}
\centering
\ \psfig{figure=\CGC/icons/sources.pal.ps}
\caption{Source stars in the CGC domain.}
\label{figure CGC source stars}
\end{figure}

Source stars have no inputs and only produce data on their outputs.
Figure\tie\ref{figure CGC source stars} shows the palette of CGC source
stars.  The following stars are equivalent to the SDF stars of the same
name (\pxref{SDF Source Stars}):
\code{Const},
\code{IIDUniform},
\code{Impulse},
\code{Ramp},
\code{singen},
\code{WaveForm}.
Stars that are unique to the CGC domain are described briefly below.

\begin{blocklist}{(CGC block)}

\block{PCMread}
Read \(\mu\)-law PCM from a file and convert to linear floating point.
The file can be the audio port \file{/dev/audio},
if supported by the workstation.

\end{blocklist}

\node CGC Sink Stars
\subsection{Sink Stars}

\begin{figure}
\centering
\ \psfig{figure=\CGC/icons/sinks.pal.ps}
\caption{Sink stars in the CGC domain.}
\label{figure CGC sink stars}
\end{figure}

Sink stars have no outputs and only consume data on their inputs.
Figure\tie\ref{figure CGC sink stars} shows the palette of CGC sink stars.
The following stars are equivalent to the SDF stars of the
same name (\pxref{SDF Sink Stars}):
\code{BlackHole},
\code{Printer},
\code{Xgraph},
\code{XMgraph},
\code{Xscope},
\code{XYgraph}.
Stars that are unique to the CGC domain are described briefly below.

\begin{blocklist}{(CGC block)}
\block{PCMwrite}
Convert input to \(\mu\)-law PCM and write it to a file.
The file can be the audio port \file{/dev/audio},
if supported by the workstation.

\block{SunSound}
Convert input to \(\mu\)-law PCM and write to the Sun audio port
\file{/dev/audio}.
The input is expected to be in the range from -1.0 to 1.0.
If the \var{aheadLimit} parameter is non-negative, then it specifies
\vindex{aheadLimit parameter}
the maximum number of samples that the program is allowed to compute
ahead of real time.

\end{blocklist}

\node CGC Arithmetic Stars
\subsection{Arithmetic Stars}

\begin{figure}
\centering
\ \psfig{figure=\CGC/icons/arithmetic.pal.ps}
\caption{Arithmetic stars in the CGC domain.}
\label{figure CGC aritmetic stars}
\end{figure}

Arithmetic stars perform simple functions.
Figure\tie\ref{figure CGC aritmetic stars} shows the palette of CGC
arithmetic stars.
All of the stars are equivalent to the SDF stars
of the same name (\pxref{SDF Arithmetic Stars}):
\code{Add},
\code{Gain},
\code{Integrator},
\code{Mpy},
\code{Sub}.

\node CGC Nonlinear Stars
\subsection{Nonlinear Stars}

\begin{figure}
\centering
\ \psfig{figure=\CGC/icons/nonlinear.pal.ps}
\caption{Nonlinear stars in the CGC domain.}
\label{figure CGC nonlinear stars}
\end{figure}

Nonlinear stars perform simple functions.
Figure\tie\ref{figure CGC nonlinear stars} shows the palette of CGC
nonlinear stars.
The following stars are equivalent to the SDF stars
of the same name (\pxref{SDF Nonlinear Stars}):
\code{conj},
\code{Cos},
\code{Exp},
\code{expjx},
\code{Log},
\code{powerEst},
\code{Quant},
\code{Reciprocal},
\code{Sin},
\code{Sqrt},
\code{Table}.
Stars that are unique to the CGC domain are described briefly below.

\begin{blocklist}{(CGC block)}

\block{Expr}
General expression evaluation.  This star evaluates the expression
given by the \var{expr} parameter and writes the result on
\var{output}.  The default expression, which is \samp{$ref(in#1)}
simply copies the first input to the output.

\block{Thresh}
Compares \var{input} values to \var{threshold}.  The \var{output} is
\samp{0} if \var{input} \( \leq \) \var{threshold}, otherwise it is
\samp{1}.

\end{blocklist}

\node CGC Control Stars
\subsection{Control Stars}

\begin{figure}
\centering
\ \psfig{figure=\CGC/icons/control.pal.ps}
\caption{Control stars in the CGC domain.}
\label{figure CGC control stars}
\end{figure}

Control stars are used for routing data and other control functions.
Figure\tie\ref{figure CGC control stars} shows the palette of CGC
control stars.
The following stars are equivalent to the SDF stars
of the same name (\pxref{SDF Control Stars}):
\code{DownSample},
\code{Fork},
\code{UpSample}.
Stars that are unique to the CGC domain are described briefly below.

\begin{blocklist}{(CGC block)}

\block{Sleep}
Suspend execution for an interval (in milliseconds).
The input is passed to the output when the process resumes.

\block{Spread}
Takes one input and produces multiple outputs.
This star does not generate any code. In multiprocessor code generation,
this star is automatically attached to a porthole whose
outputs are passed to more than one destination (one ordinary block and
one \code{Send} star, more than one \code{Send} star, and so on.)

\block{Collect}
Takes multiple inputs and produces one output.
This star does not generate code. In multiprocessor code generation,
it is automatically attached to a porthole if it has multiple sources
Its role is just opposite to that of \code{Spread} star.

\end{blocklist}

\node CGC Conversion Stars
\subsection{Conversion Stars}

\begin{figure}
\centering
\ \psfig{figure=\CGC/icons/conversion.pal.ps}
\caption{Type-conversion stars in the CGC domain.}
\label{figure CGC conversion stars}
\end{figure}

Conversion stars are used to convert between complex and real numbers.
Figure\tie\ref{figure CGC conversion stars} shows the palette of CGC
conversion stars.
All of the stars are equivalent to the SDF stars
of the same name (\pxref{SDF Conversion Stars}):
\code{CxToRect},
\code{PolarToRect},
\code{RectToCx},
\code{RectToPolar}.

\node CGC Signal Processing Stars
\subsection{Signal Processing Stars}

\begin{figure}
\centering
\ \psfig{figure=\CGC/icons/dsp.pal.ps}
\caption{Signal processing stars in the CGC domain.}
\label{figure CGC dsp stars}
\end{figure}

Figure\tie\ref{figure CGC dsp stars} shows the palette of CGC
signal processing stars.
All of the stars are equivalent to the SDF stars
of the same name (\pxref{SDF Signal Processing Stars}):
\code{DB},
\code{FFTCx},
\code{FIR},
\code{LMS},
\code{LMSTkPlot}.

\node CGC BDF Stars
\subsection{BDF Stars}


\begin{figure}
\centering
\ \psfig{figure=\CGC/icons/bdf.pal.ps}
\caption{BDF stars in the CGC domain.}
\label{figure CGC BDF stars}
\end{figure}

BDF stars are used for conditionally routing data.
Figure\tie\ref{figure CGC BDF stars} show the palette of BDF stars in
the CGC domain.  These stars require the use of the \code{bdf-CGC}
target (\pxref{CGC Single-Processor Targets}).
\tindex{bdf-CGC target}

\begin{blocklist}{(CGC Block)}
\block{Select}
This star requires a BDF scheduler.
If the value on the \var{control} line is nonzero, \var{trueInput}
is copied to the \var{output}; otherwise, \var{falseInput} is.

\block{Switch}
This star requires a BDF scheduler.
Switches \var{input} events to one of two outputs, depending on
the value of the \var{control} input.  If \var{control} is true, the
value is written to \var{trueOutput}; otherwise it is written to
\var{falseOutput}.

\end{blocklist}

\node CGC Tcl/Tk Stars
\subsection{Tcl/Tk Stars}

\begin{figure}
\centering
\ \psfig{figure=\CGC/tcltk/icons/tcltk.pal.ps}
\caption{Tcl/Tk stars in the CGC domain.}
\label{figure CGC tcltk stars}
\end{figure}

Tcl/Tk stars require the use of the \code{TclTk_Target} target.  They
can be used to provide an interactive user interface with Tk widgets.
Figure\tie\ref{figure CGC tcltk stars} shows the palette of Tcl/Tk
stars available in the CGC domain.  Short descriptions of the stars are
given below.

\begin{blocklist}{(CGC block)}

\block{TkEntry}
Output a constant signal with value determined by a Tk entry box.

\block{TkImpulse}
Output an impulse when a button is pushed.

\block{TclScript}
Sends input values to a Tcl script.  Gets output values from a Tcl script.
The star can communicate with Tcl either synchronously or asynchronously.

\block{TkSunSound}
Just like \code{SunSound} (\pxref{CGC Sink Stars}),
except that a Tk slider is put in the master
control panel to control the volume.

\end{blocklist}

\node CGC Demo Overview
\section{An Overview of CGC Demos}

\begin{figure}
\centering
\ \psfig{figure=\CGC/demo/init.pal.ps}
\caption{Top-level palette of demos in the CGC domain.}
\label{figure CGC demos}
\end{figure}

Figure\tie\ref{figure CGC demos} shows the top-level palette of CGC
demos.  The demos are divided into categories:  basic, multirate,
signal processing, sound, Tcl/Tk, and BDF demos.  Many of the demos in
the CGC domain have equivalent counterparts in the SDF or BDF domains.
\xref{SDF Demo Overview} or \nxref{BDF Demo Overview} for brief
descriptions of these demos.  Brief descriptions of the demos unique to
the CGC domain are given in the following sections.  Refer to the Star
Atlas for more detailed descriptions.

\node CGC Basic Demos
\subsection{Basic Demos}

\begin{figure}
\centering
\ \psfig{figure=\CGC/demo/basic.pal.ps}
\caption{Basic CGC demos.}
\end{figure}

The following demos are equivalent to the SDF demos
of the same name (\pxref{SDF Demo Overview}):
\code{butterfly},
\code{chaos},
\code{integrator},
\code{quantize}.
The other demos in this palette are described briefly below.

\begin{blocklist}{(CGC Demo)}

\block{nonlinear}
This simple system plots four nonlinear functions over the range
1.0 to 1.99.  The four functions are exponential, natural logarithm,
square root, and reciprocal.

\block{typeConversion}
This simple demo tests the automatic splicing in of type conversion stars.
It only generates code and does not run.

\end{blocklist}

\node CGC Multirate Demos
\subsection{Multirate Demos}

\begin{figure}
\centering
\ \psfig{figure=\CGC/demo/multirate.pal.ps}
\caption{Multirate CGC demos.}
\end{figure}

The following demos are equivalent to the SDF demos
of the same name (\pxref{SDF Demo Overview}):
\code{interp},
\code{filterBank}.
The other demos in this palette are described briefly below.

\begin{blocklist}{(CGC Demo)}

\block{upsample}
This simple up-sample demo tests static buffering.
Each invocation of the
\code{XMgraph}
star reads its input from a fixed buffer location
since the buffer between the
\code{UpSample}
star and the
\code{XMgraph}
star is static.

\block{loop}
This demo demonstrates the code size reduction achieved with a loop-generating
scheduling algorithm.

\end{blocklist}

\node CGC Signal Processing Demos
\subsection{Signal Processing Demos}

\begin{figure}
\centering
\ \psfig{figure=\CGC/demo/dsp.pal.ps}
\caption{Signal processing demos in the CGC domain.}
\end{figure}

The following demos are equivalent to the SDF demos
of the same name (\pxref{SDF Demo Overview}):
\code{adaptFilter},
\code{dft}.
The \code{animatedLMS} demo is described in \nxref{CGC Tcl/Tk Demos}.

\node CGC Sound-Making Demos
\subsection{Sound-Making Demos}

\begin{figure}
\centering
\ \psfig{figure=\CGC/demo/sound.pal.ps}
\caption{Sound-making demos in the CGC domain.}
\label{figure CGC sound demos}
\end{figure}

This palette contains only one demo.  Your workstation must be equipped
with an audio device that can accept \(\mu\)-law encoded PCM for
this demo to work.  Some other sound making demos can be found in the
Tcl/Tk palette. \xref{CGC Tcl/Tk Demos}.

\begin{blocklist}{(CGC Demo)}

\block{sound}
Generate a sound to play over the workstation speaker.

\end{blocklist}

\node CGC Tcl/Tk Demos
\subsection{Tcl/Tk Demos}

\begin{figure}
\centering
\ \psfig{figure=\CGC/tcltk/demo/init.pal.ps}
\caption{Tcl/Tk demos in the CGC domain.}
\label{figure CGC tcltk demos}
\end{figure}

These demos show off the capabilities of the Tcl/Tk stars, which must
be used with the \code{TclTk_Target} target.
\tindex{TclTk_Target target}
Graphical user interface widgets are used to control input parameters
and to produce animation.  Many of these demos also produce sound on the
workstation speaker with the \code{TkSunSound} star (\pxref{CGC Tcl/Tk Stars}).
\tindex{TkSunSound (CGC Block)}
Figure\tie\ref{figure CGC tcltk demos} shows the demos that are available.

\begin{blocklist}{(CGC Demo)}

\block{tremolo}
This demo produces a tremolo (amplitude modulation) effect  on the
workstation speaker.  You can adjust the pitch, modulation
frequency, and volume in real time.

\block{fm}
This demo uses frequency modulation (FM) to synthesize a tone on the
workstation speaker.  You can adjust the modulation index, pitch, and
volume in real time.

\block{animatedLMS}
This demo is a simplified version of the SDF demo of the same name
(\pxref{SDF Demo Overview}).

\block{impulse}
This demo generates tones on the workstation speaker with decaying
amplitude envelopes using frequency modulation synthesis.  You can make
tones by pushing a button. You can adjust the pitch, modulation index,
and volume in real time.

\block{synth}
This demo generates sinusoidal tones on the workstation speaker.  You
can control the pitch with a piano-like interface.

\block{dtmf}
This demo generates the same dual-tone multiple-frequency tones you
hear when you dial your phone.  The interface resembles the keypad of a
telephone.

\block{scriptTest}
This demo show the use of several kinds of Tk widgets for user input.
Push buttons generate tones or noise, and sliders adjust the frequency
and volume in real time.

\block{ball}
This demo exhibits sinusoidal motion with a ball moving back and forth.

\block{ballAsync}
This demo is the same as the \code{ball} demo except that animation is
updated asynchronously.

\block{universe}
This demo shows the movements of the Sun, Venus, Earth, and Mars in a
Ptolemaic (Earth centered) universe.

\end{blocklist}

\node CGC BDF Demos
\subsection{BDF Demos}

\begin{figure}
\centering
\ \psfig{figure=\CGC/demo/bdf.pal.ps}
\caption{BDF demos in the CGC domain.}
\label{figure CGC BDF demos}
\end{figure}

Figure\tie\ref{figure CGC BDF demos} shows the palette of universes
that demonstrate the use of BDF stars in the
CGC domain.  The \code{timing} demo is equivalent to the BDF demo of
the same name.  The demos \code{bdf-if} and \code{bdf-doWhile} are
equivalent to the BDF demos named \code{ifThenElse} and \code{loop}.
\xref{BDF Demo Overview} for short descriptions of these demos.

