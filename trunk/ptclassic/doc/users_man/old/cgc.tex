\c Version $Id$
\node CGC Domain
\chapter{CGC Domain}

\Author{J. Buck \\
S. Ha \\
E. A. Lee \\
T. M. Parks}
\date{$Date$}

The \dfn{CGC domain} of \Ptolemy\ generates code
for the \code{C} programming language.
\cindex{code generation in C}
\cindex{C code generation}
\cindex{CGC domain}
This domain supports both synchronous dataflow
(\pxref{SDF Domain})
\cindex{synchronous dataflow}
\cindex{dataflow, synchronous}
and boolean-controlled dataflow
(\pxref{BDF Domain})
\cindex{boolean-controlled dataflow}
\cindex{dataflow, boolean-controlled}
models of computation.  The model of
computation assocaited with a particular program graph is determined by
which target is selected.  The \code{bdf-CGC} target supports the BDF
model, while all other targets in the CGC domain support the more
limited SDF model.  Code can be generated for single-processor machines
and multi-processor machines.  The targets that support
single-processors include \code{default-CGC}, \code{TclTk_Target}, and
\code{bdf-CGC}.  The multi-processor target is \code{unixMulti_C}.

For an overview of CGC stars and demos, please refer to the star atlas.

\node Single-Processor Targets
\section{Single-Processor Targets}

The single-processor targets of the CGC domain
generate \code{C} code from dataflow program graphs.
All code is written
to and compiled and run on the computer specified by the \var{host}
parameter.  If a remote computer is specified by \var{host} then
\code{rsh} commands will be used to place files on that compuer and to
invoke the compiler.  You should verify that your \file{.rhosts} file
is properly configured so that \code{rsh} will work.
The code is placed in the directory specified by the \var{directory}
parameter, in files named according to the \var{file} parameter.  The
command used to compile the code is determined by the
\var{compileCommand}, \var{compileOptions}, and \var{linkOptions}
parameters.

The parameters \var{display?}, \var{compile?}, and \var{run?} are
provided for debugging and development.  For example, during
development you may want to set \var{display?} to \samp{TRUE} and set
\var{compile?} to \samp{FALSE}.  This would allow you to view the
generated code without compiling it.  Later you may want to set
\var{compile?} back to \samp{TRUE} but set \var{run?} to
\samp{FALSE}.  This would allow you to
verify that the generated code compiles correctly
without attempting to run the program.  (If compilation
fails, then the program will not be run regardless of the state of
\var{run?}.)  Once everything works, you may want to set
\var{display?} to \code{FALSE} to avoid the annoyance of having the code
displayed for every run.

The complete list of parameters common to the single-processor targets
of the CGC domain is given below.

\begin{statelist}
\state{host}{STRING}{}
The name of the computer where programs will be compiled and run.  If
the name is blank, which is the default, then programs will be compiled
and run on the same computer where \Ptolemy\ is running.

\state{directory}{STRING}{PTOLEMY_SYSTEMS}
The destination directory for files created in the code
generation process.  If the directory does not already exist,
it will be created.  All files are written into this directory on the
\var{host} computer.

\state{file}{STRING}{}
The base name for files created in the code generation process.
If the name is blank, which is the default, then a suitable name will be
generated automatically.
The \file{.c} extension is automatically added to this name to form the
complete file name.

\state{display?}{INT}{TRUE}
Controls the display of the generated code.

\state{compile?}{INT}{TRUE}
Controls the invokation of the compiler after code generation.

\state{run?}{INT}{TRUE}
Controls the running of the program after compilation.

\state{staticBuffering}{INT}{TRUE}
If \samp{TRUE}, then attempt to static, compile-time addressing of data
buffers between stars.  Otherwise, use dynamic, run-time addressing.

\state{funcName}{STRING}{main}
The name of the main function.  The default value of \samp{main} is suitable
for generating stand-alone programs.  Choose another name if you wish to use
the generated code as a procedure that is called from your own main program.

\state{compileCommand}{STRING}{cc}
Command name of the compiler.

\state{compileOptions}{STRING}{}
Options passed to the compiler.

\state{linkOptions}{STRING}{-lm}
Options passed to the linker.

\state{resources}{STRINGARRAY}{STDIO}
List of abstract resources that the host computer has.

\end{statelist}

\node default-CGC
\subsection{\protect\code{default-CGC}}

The \code{default-CGC} target generates \code{C} code for a single computer
from a SDF program graph.  In addition to the common parameters listed in the
previous section, this target has the following parameters.

\begin{statelist}
\state{Looping Level}{INT}{0}
Determines which scheduling algorithm is used.  If set to \samp{0}, an
in-line schedule will be generated.  If set to \samp{1}, then a
conservative loop-generating algorithm is used.  If set to \samp{2}, then a
more exhaustive loop-generating algorithm is used.  The loop-generating
algorithms
generally produce more compact code but may take longer to run.
\end{statelist}

\node TclTk_Target
\subsection{\protect\code{TclTk_Target}}

This target, which is derived from the \code{default-CGC} target
(\pxref{default-CGC}), must be used when Tcl/Tk stars are present in
the program graph.  The parameters differ from those of the
\code{default-CGC} target only in their default values.

\begin{statelist}
\state{loopingLevel}{INT}{1}

\state{funcName}{STRING}{go}

\state{compileOptions}{STRING}{-I$PTOLEMY/tcl/tk/src \\
    -I$PTOLEMY/tcl/tcl/src \\
    -I$PTOLEMY/src/domains/cgc/tcltk/lib}

\state{linkOptions}{STRING}{-L$PTOLEMY/tcl/tk/lib/$ARCH.opt \\
    -L$PTOLEMY/tcl/tcl/lib/$ARCH.opt \\
    -L$PTOLEMY/lib.$ARCH \\
    -L$PTOLEMY/tcl/tk/lib/$ARCH.opt \\
    -L/usr/X11/lib -ltk -ltcl -lptk -lXpm -lX11 -lm}

\end{statelist}

\node bdf-CGC
\subsection{\protect\code{bdf-CGC}}

This target supports the boolean-controlled dataflow (BDF) model of
computation.  It must be used when BDF stars are present in the program
graph.  It can also be used with program graphs that contain only
synchronous dataflow (SDF) stars.

The \code{bdf-CGC} target has the same parameters as the \code{default-CGC}
target (\pxref{default-CGC})
with the exception that the \var{Looping Level} parameter is absent.

\node Multi-Processor Targets
\section{Multi-Processor Targets}

\node unixMulit_C
\subsection{\protect\code{unixMulti_C}}

\node CGCcm5
\subsection{\protect\code{CGCcm5}}

\begin{ignore}

\node Star Overview
\section{An Overview of CGC Stars}

\node Source Stars
\subsection{Source Stars}

\begin{description}
\item[Const]
\item[Ramp]
\item[IIDUniform]
\item[WaveForm]
\item[PCMread]
\item[singen]
\item[TkSlider]
\item[TkEntry]
\item[TkImpulse]
\end{description}

\node Sink Stars
\subsection{Sink Stars}

\begin{description}
\item[BlackHole]
\item[Xgraph]
\item[XMgraph]
\item[Xscope]
\item[XYgraph]
\item[Printer]
\item[PCMwrite]
\item[SunSound]
\end{description}

\node Arithmetic Stars
\subsection{Arithmetic Stars}

\node Nonlinear Stars
\subsection{Nonlinear Stars}

\node Control Stars
\subsection{Control Stars}

\node Conversion Stars
\subsection{Conversion Stars}

\node Signal Processing Stars
\subsection{Signal Processing Stars}

\node Dynamic Dataflow Stars
\subsection{Dynamic Dataflow Stars}

\node Boolean-Controlled Dataflow Stars
\subsection{Boolean-Controlled Dataflow Stars}

\node Tcl/Tk Stars
\subsection{Tcl/Tk Stars}

\node Demo Overview
\section{An Overview of CGC Demos}

\end{ignore}
